\documentclass[a4paper,11pt]{article}
\usepackage[utf8]{inputenc} % un package
\usepackage[francais]{babel} %active le mode francais
\usepackage[top=2cm , bottom=2cm , left=2cm , right=2cm]{geometry} %propriétés de notre page
\usepackage{amsmath} %liste de symboles et applications mathématiques
\usepackage{color} %Permet d'utiliser la couleur dans nos documents
\usepackage{listings} %Paquet de coloration syntaxique (langages)
\usepackage{hyperref} % Créer des liens et des signets 

% Informations du rapport
\title {Rapport \\ Travaux Pratiques Réseaux (Ethernet)}
\author {Quentin Tonneau - Adrien Lardenois}
\date{}
%Propriétés des liens
\hypersetup{
colorlinks=true, %colorise les liens  
urlcolor= blue, %couleur des hyperliens 
linkcolor= blue,%couleur des liens internes 
} 

\begin{document}
	\maketitle %insère l'en-tête du rapport
	\tableofcontents %insère la table des matières ATTENTION : Compiler deux fois en cas de changements
	\newpage % Nouvelle page
	
	
	
	
	
	
	\section{Introduction}
	Après avoir mis en réseau un parc de pc (cf rapport tp \no 1), nous nous intéressons maintenant à la communication entre ces derniers. En s'appuyant sur le cours de réseaux Ethernet et nos connaissances en langage C dans l'environnement Linux, nous allons concevoir une suite d'applications destinées à interagir avec les PC voisins. Une bibliothèque de création et d'envoi de trame, ainsi qu'un sniffeur de réseau (affiche l'ensemble des trames circulant au voisinage de notre matériel) nous sont fournis afin de franchir les couches non étudiées en classe. %@TODO compléter les couches du modèle
	\subsection{Écouter le réseau}
	Avant de communiquer sur un réseau, il nous faut un certain nombre d'informations sur le matériel avec lequel on souhaite ``entrer en discussion'', c'est à dire :
	%liste à puce
	\begin{itemize}
		\item Le nombre de personnes présentes sur le réseau
		\item Leurs adresses
		\item Le protocole de communication
		\item La nature des messages (émetteur, destinataire, type de message)
	\end{itemize}
	Pour cela, nous concevons un programme qui filtre les trames circulant sur le réseau, en ne conservant que les trames de type 9000 dont nous sommes le destinataire, puis affiche le message en question,tout en dressant une liste de toutes les machines (adresses MAC) présentes sur le réseau. On pourrait associer ce programme à un module de conversation type IRC\footnote{Internet Relay Tchat}, qui n'affiche que les messages personnels ou à destination de l'ensemble des utilisateurs.
	\subsection{Envoyer un message}
	Après avoir pris connaissance des appareils connectés à notre pc, nous pouvons maintenant écrire un programme d'envois de message.
	\subsection{Accusé la réception d'un message}
	blabla
	\section{Réception des trames}
	La reception de trames se fait par redirection de la sortie du sniffer vers notre programme défini par \textbf{recevoir\_trame.c}, au moyen d'un pipeline.
	Les informations reçues sont traitées une à une par la fonction \textit{lire\_trame()}. On vérifie alors si l'adresse de destination et l'adresse source nous sont déjà connus puis, si l'un et/ou l'autre est ``nouveau'', on l'ajoute à notre liste des matériels connectés.

	Nous avons choisi de représenter la liste des adresses rencontrées par une liste chainée. Cela a pour avantage de permettre l'ajout d'un maillon sans connaitre la taille actuelle de la liste, au contraire d'un tableau qu'il aurai fallu redimensionner au fur et à mesure.
	
	Dans un second temps, on vérifie si la trame appartient au type défini spécialement pour le TP (9000). Si c'est le cas, on affiche à l'écran les informations suivantes :
	\begin{itemize}
		\item Numéro de la trame (au moyen d'un compteur)
		\item Adresse source
		\item Contenu de la trame
	\end{itemize}

	Enfin, chaque fois que Y trames ont été affichées \footnote{Y étant défini au début du programme}, on affiche un récapitulatif des adresses qui circulent sur le réseau. Suivant le choix de Y (50 dans nos exemples) et en fonction du traffic sur le reseau, on beneficie alors d'un affichage clair des trames qui circulent avec un bilan regulier des utilisateurs presents. %d'ailleurs on pourrai de temps en temps vider la liste pour se debarasser des utilisateurs inactifs sur le reseau depuis trop longtemps.
	\subsection{la fonction lire\_trame}
	La fonction \textit{lire\_trame} est définie dans \textbf{fonctions.c}. Elle retire les 3 premiers caractères de la sortie du sniffer qui ne nous intéressent pas, puis récupère la longueur de la trame. Apres quoi elle capte toute la trame grâce à la fonction \textit{get\_buf} et la retourne en sortie.
	\subsection{la fonction recherche}
	La fonction \textit{ recherche} parcoure la liste chainée à partir du premier maillon et s'arrête si en passant au suivant elle arrive à NULL. Il est possible de faire le test à la fin car l'initialisation de la liste garanti l'existence d'au moins un maillon different de NULL.

	Pour chaque maillon, on compare le champ adresse à celle que l'on cherche : s'il y a correspondance on sort directement en renvoyant 1, sinon on passe au suivant. Si la recherche a eu lieu jusqu'au bout sans succes, on retourne 0.
	\subsection{la fonction ajout}
	La fonction d'ajout ajoute un élément en tete de liste. On crée simplement un nouvel élément que l'on défini comme l'adresse à ajouter et dont le suivant est l'ancienne tete de liste.
	\subsection{la fonction affiche}
	La fonction  \textit{affiche} parcoure la liste de la même maniere que  \textit{recherche} . Pour chaque étape de la liste, elle affiche l'adresse et separe les adresses par :::
	\section{Envois des trames (type 9000)}
	\section{Synthèse et automatisation}
	\section{Bilan}
\end{document}
